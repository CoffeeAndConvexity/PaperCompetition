
\section{Equilibrium and Duality} \label{sec:equi-and-dual}

% Reorganize: new order of results in main body (but proofs might go back and forth)
% \begin{itemize}
% 	\item ``Primal'' and ``dual'' sup/inf attainments; ``primal'' also attainable via pure solution.
% 	\item Weak duality: given primal and dual feasible solutions, we have ${\rm pobj} + C \leq {\rm dobj}$.
% 	\item Strong duality: exists primal and dual opt solutions that match.
% 	\item ME iff both optimal.
% \end{itemize}

Due to intrinsic limitations of general infinite-dimensional convex optimization duality theory, we cannot start with a convex program and then derive its dual.
Instead, we directly \emph{propose} two infinite-dimensional convex programs, and then proceed to show from first principles that they exhibit optimal solutions and a strong-duality-like relationship.
First, we give a direct generalization of the (finite-dimensional) Eisenberg-Gale convex program \citep{eisenberg1961aggregation,nisan2007algorithmic}:
\begin{align}
z^* = \sup_{x\in (L^\infty(\Theta)_+)^n} \sum_i B_i \log \langle v_i, x_i \rangle\ \ {\rm s.t.}\ \sum_i x_i \leq \ones. \tag{$\mathcal P_{\rm EG}$} \label{eq:eg-primal}
\end{align}

Motivated by the dual of the finite-dimensional EG convex program \cite[Lemma 3]{cole2017convex}, we also consider the following convex program:
\begin{align}
\begin{split}
w^* = \inf_{p\in L^1(\Theta)_+,\, \beta\in \RR^n_+} \left[ \langle p, \ones \rangle - \sum_i B_i \log \beta_i\right]  \quad {\rm s.t.} \ p \geq \beta_i v_i\ \almeve, \forall\,i. 
\end{split} \tag{$\mathcal D_{\rm EG}$}
\label{eq:eg-dual-beta-p}
\end{align}

We state our central theoretical results in the following theorem. Parts of this theorem are stated in more detail in subsequent lemmas and theorems. 
Proofs of all theoretical results can be found in the appendix. 
\begin{theorem}
	\begin{enumerate*}[label=(\alph*)]
		\item The supremum $z^*$ of \eqref{eq:eg-primal} is attained via a \emph{pure} optimal solution $x^*$, that is, $x^* = (x^*_i)$ and $x^*_i = \ones_{\Theta_i}$ for a.e.-disjoint measurable subsets $\Theta_i\subseteq \Theta$. \label{item:eg-primal-attain}
		\item The infimum $w^*$ of \eqref{eq:eg-dual-beta-p} is attained via an optimal solution $(p^*, \beta^*)$, in which $\beta^*\in \RR^n_+$ is unique and $p^* = \max_i \beta^*_i v_i$ a.e. \label{item:eg-dual-p-beta-attain}
		\item A pair of allocations and prices $(x^*, p^*) \in (L^\infty(\Theta)_+)^n \times L^1(\Theta)_+$ is a ME if and only if $x^*$ is an optimal solution of \eqref{eq:eg-primal} and $(p^*, \beta^*)$ is the (a.e.-unique) optimal solution of \eqref{eq:eg-dual-beta-p}. \label{item:eg-equi-iff-opt}
	\end{enumerate*}
	\label{thm:eg-equi-opt-combined}
\end{theorem}

\begin{remark}
	\normalfont
	If we view \eqref{eq:eg-dual-beta-p} as the primal, then it can be shown that its Lagrange dual is \eqref{eq:eg-primal}, and weak duality follows (see, e.g., \cite[\S 3]{ponstein2004approaches}). 
	However, we cannot conclude strong duality, or even primal or dual optimum attainment, since $L^1(\Theta)_+$ has an empty interior \cite[\S 8.8 Problem 1]{luenberger1997optimization} and hence Slater's condition does not hold.
	If we choose the space for valuations and prices to be $L^\infty(\Theta)$ instead of $L^1(\Theta)$ for the space of allocations $x_i$ (i.e., the underlying Banach space of \eqref{eq:eg-primal}), then \eqref{eq:eg-dual-beta-p}, with $p \in L^\infty(\Theta)_+$, does satisfy Slater's condition \cite[\S 8.8 Problem 2]{luenberger1997optimization}. 
	However, its dual is \eqref{eq:eg-primal} but with the nonnegative cone $L^\infty(\Theta)_+$ (in which each $x_i$ lies) replaced by the (much larger) cone $\{ g \in L^\infty(\Theta)^*: \langle f, g \rangle \geq 0,\, \forall\, f\in L^\infty(\Theta)_+ \}\subseteq L^\infty(\Theta)^*$. In this case, not every bounded linear functional $g\in L^\infty(\Theta)$ can be represented by a measurable function $\tilde{g}$ such that $\langle f, g \rangle = \int_\Theta \tilde{g}f d\mu$ (see, e.g., \citep{day1973normed}). 
	Therefore, we still cannot conclude that \eqref{eq:eg-primal} has an optimal solution in $(L^1(\Theta)_+)^n$ satisfying strong duality. 
	Similar issues occur when \eqref{eq:eg-primal} is viewed as the primal instead. \label{remark:why-define-cp-then-prove}
\end{remark}

We briefly explain the proof ideas of Theorem~\ref{thm:eg-equi-opt-combined}. 
Unlike the finite-dimensional case, the feasible region of \eqref{eq:eg-primal} here, although being closed and bounded in the Banach space $L^\infty(\Theta)$, is not compact. In fact, it is easy to construct an infinite sequence in the feasible region that does not have any convergent subsequence.
This issue can be circumvented using the following lemma.
\begin{lemma}
	Define the set of feasible utilities as\\
	$U = U(v, \Theta) = \left\{ (u_1, \dots, u_n): u_i = \langle v_i, x_i\rangle,\, x\in (L^\infty(\Theta)_+)^n,\, \sum_i x_i \leq \ones \right\} \subseteq \RR_+^n \label{eq:def-U-U(v,Theta)}$\\
	and the set of utilities attainable via pure allocations as\\
	$U' = U'(v, \Theta) = \left\{ (v_1(\Theta_1), \dots, v_n(\Theta_n)): \Theta_i \subseteq \Theta\, \text{measurable and a.e.-disjoint} \right\} \subseteq \RR_+^n$.\\
	Then, $U = U'$ and this set is convex and compact.
	\label{lemma:U-convex-compact}
\end{lemma}
For part \ref{item:eg-primal-attain}, using Lemma~\ref{lemma:U-convex-compact}, we can show that there exists $u^* \in \RR_{++}^n$ such that $z^* = \sum_i B_i \log u^*_i$, which then ensures that $z^*$ is attained by some \emph{pure} feasible solution $\{\Theta_i\}$ of \eqref{eq:eg-primal}, that is, $\Theta_i\subseteq \Theta$ are a.e.disjoint and $v_i(\Theta_i) = u^*_i$. Part \ref{item:eg-dual-p-beta-attain} follows by reformulating \eqref{eq:eg-dual-beta-p} into a finite-dimensional convex program in $\beta\in \RR_+^n$. 
For a  fixed $\beta>0$, setting $p = \max_i \beta_i v_i$ clearly minimizes the objective of \eqref{eq:eg-dual-beta-p}. 
Since $\beta\geq 0$ and $v_i \in L^1(\Theta)_+$, we have
$ 0 \leq \max_i \beta_i v_i \leq \|\beta\|_1 \sum_i v_i$,
where the right-hand side is $L^1$-integrable since each $v_i$ is. Hence, $\max_i \beta_i v_i \in L^1(\Theta)_+$ as well. 
Thus we can also reformulate \eqref{eq:eg-dual-beta-p} as the following convex program:
\begin{align}
\inf_{\beta\in \RR_+^n} \left[\left \langle \max_i \beta_i v_i, \ones \right \rangle - \sum_i B_i \log \beta_i\right].  \label{eq:eg-dual-beta-1}
\end{align} 
\begin{lemma}
	The infimum of \eqref{eq:eg-dual-beta-1} is attained via a unique minimizer $\beta^* > 0$. The optimal solution $(p^*, \beta^*)$ of \eqref{eq:eg-dual-beta-p} has a unique $\beta^*$ and satisfies $p^* = \max_i \beta^*_i v_i$ a.e.
	\label{lemma:beta-dual-attain}
\end{lemma}
To show Part \ref{item:eg-equi-iff-opt}, we first establish weak duality and KKT conditions in the following lemma. 
As mentioned before, this is necessary due to the lack of general duality results in infinite-dimensional convex optimization.
These conditions parallel those in KKT-type optimality conditions in classical nonlinear optimization over Euclidean spaces (see, e.g., \citet[\S 3.3.1]{bertsekas1999nonlinear}).
\begin{lemma}
	Let $C = \|B\|_1 - \sum_i B_i \log B_i$. The following hold.
	\begin{enumerate*}[(a)]
		\item Weak duality: $C + z^* \leq w^*$. \label{item:eg-weak-duality}
		\item KKT conditions: For $x^*$ feasible to \eqref{eq:eg-primal} and $(p^*, \beta^*)$ feasible to \eqref{eq:eg-dual-beta-p}, they are both optimal (i.e., attaining the optima $z^*$ and $w^*$ respectively) if and only if \label{item:eg-KKT-iff}
	% \item Suppose $x^*$ is feasible to \eqref{eq:eg-primal}, $(p^*, \beta^*)$ is feasible to \eqref{eq:eg-dual-beta-p} and \eqref{eq:mkt-clear}, \eqref{eq:u*=B/beta*} hold. Then, $x^*$ and $(p^*, \beta^*)$ are optimal to \eqref{eq:eg-primal} and \eqref{eq:eg-dual-beta-p}, respectively. Meanwhile, $C + z^* = w^*$. \label{item:eg-KKT-backward-onlyif}
	\end{enumerate*} 
	\begin{align}
		\left \langle p^*, \ones - \sum_i x^*_i\right \rangle = 0, \ \ 
		\langle v_i, x_i^*\rangle = u^*_i :=  \frac{B_i}{\beta^*_i}, \ \forall\, i, \ \ 
		\langle p^* - \beta^*_i v_i, x^*_i \rangle = 0,\ \forall\, i.
		\label{eq:kkt-mkt-clear-etc}
	\end{align} 
	\label{lemma:weak-duality}
\end{lemma}
Thus, in spite of the general difficulties with duality theory in infinite dimensions, \eqref{eq:eg-primal} and \eqref{eq:eg-dual-beta-p} behave like duals of each other: strong duality holds, and KKT conditions hold if and only if a pair of feasible solutions are both optimal. In \eqref{eq:kkt-mkt-clear-etc}, the first condition is market clearance; the second condition says each buyer $i$ pays $\beta^*_i$ for each unit of utility she receives. As such, $\beta^*$ is known as the (equilibrium) \emph{utility price}. Since $p^* \geq \beta^*_i v_i$, we know that under prices $p^*$, buyer $i$ must pay at least $\beta^*_i$ for each unit of utility. Using Lemma \ref{lemma:weak-duality}, we can show the following theorem, which is an expanded version of Part \ref{item:eg-equi-iff-opt} of Theorem~\ref{thm:eg-equi-opt-combined} regarding the equivalence of market equilibrium and optimality w.r.t. the convex programs.
\begin{theorem}
	Assume $x^*$ and $(p^*, \beta^*)$ are optimal solutions of \eqref{eq:eg-primal} and \eqref{eq:eg-dual-beta-p}, respectively. 
	Then $(x^*, p^*)$ is a ME, $\langle p^*, x^*_i \rangle = B_i$ for all $i$, and the equilibrium utility of buyer $i$ is $u^*_i = \langle v_i, x^*_i \rangle = \frac{B_i}{\beta^*_i}$. 
	Conversely, if $(x^*, p^*)$ is a ME, then $x^*$ is an optimal solution of \eqref{eq:eg-primal} and $(p^*, \beta^*)$, where $\beta^*_i := \frac{B_i}{\langle v_i, x^*_i\rangle}$, is an optimal solution of \eqref{eq:eg-dual-beta-p}.
	\label{thm:eg-gives-me}
\end{theorem}
% The above corollary ensures that we can construct an optimal solution of \eqref{eq:eg-dual-beta-p} from a pure optimal solution of \eqref{eq:eg-primal}, which satisfy all predicates in Lemma~\ref{lemma:weak-duality}.
% The connection of Corollary~\ref{cor:eg-pure-solution-gives-strong-duality} with the notion of fair division in \cite{weller1985fair} will be discussed latter (see Corollary~\ref{cor:Bi-fair-division-weller}). 
We list some direct consequences of the results we have obtained so far. First, we give a direct consequence of Theorem~\ref{thm:eg-gives-me} and Part \ref{item:eg-weak-duality} of Lemma~\ref{lemma:weak-duality} on the structural properties of a market equilibrium; next, given a pure optimal solution $\{\Theta_i\}$ of \eqref{eq:eg-primal}, we can construct the (a.e.-unique) optimal solution $(p^*, \beta^*)$ of \eqref{eq:eg-dual-beta-p}.
\begin{corollary}
	Let $(x^*, p^*)$ be a ME. 
	Then, $x^*$ and $(p^*, \beta^*)$, where $\beta^*_i := \frac{B_i}{\langle v_i, x^*_i\rangle}$, satisfy \eqref{eq:kkt-mkt-clear-etc}. In particular, $\langle p^*, \ones - \sum_i x^*_i \rangle = 0$ shows that a buyer's equilibrium allocation $x^*_i$ must be zero a.e. outside its ``winning'' set of items $\{p^* = \beta^*_i v_i \}$.
	\label{cor:me-structiral-properties}
\end{corollary}
% \yuan{
	% Intuitively, if there is a subset $A\subseteq\Theta_i$ on which $p^* > \beta^*_i v_i$, then, items in $A$ bring utility $v_i(A)$ but incur cost $p^*(A)$, with a ratio $\frac{p^*(A)}{v_i(A)} > \beta^*_i$.
\begin{corollary}
	Let $\{\Theta_i\}$ be a pure optimal solution of \eqref{eq:eg-primal}, $u^*_i = v_i(\Theta_i)$ and $\beta^*_i = \frac{B_i}{u^*_i}$.
	\begin{enumerate*}[label=(\alph*)]
		\item For each $i$, we have $\beta^*_i v_i \geq \beta^*_j v_j$ a.e. for all $j\neq i$ on $\Theta_i$. \label{item:thm-eg-pure-solution-gives-dual-feas}
		\item Let $p^* := \max_i \beta^*_i v_i$. Then, $p^*(A) = \sum_i \beta^*_i v_i(A\cap \Theta_i)$ for any measurable set $A\subseteq \Theta$.
		\label{item:thm-eg-pure-solution-gives-p*}
		\item The constructed $(p^*, \beta^*)$ is an optimal solution of \eqref{eq:eg-dual-beta-p} and satisfies \eqref{eq:kkt-mkt-clear-etc}. 
		\label{item:thm-eg-pure-solution-gives-dual-opt}
	\end{enumerate*}
	 \label{cor:eg-pure-solution-gives-strong-duality}
\end{corollary}
% Given a pure allocation, we can also verify whether it is an equilibrium allocation using the following corollary.
% \begin{corollary}
% 	A pure allocation $\{\Theta_i\}$ is an equilibrium allocation (with equilibrium prices $p^*$) if and only if the following conditions hold with $\beta^*_i := \frac{B_i}{v_i(\Theta_i)}$ and $p^* := \max_i \beta^*_i v_i$.
% 	\begin{enumerate}
% 		\item Prices of items in $\Theta_i$ are given by $\beta^*_i v_i$: $p^* = \beta^*_i v_i$ on each $\Theta_i$, $i\in [n]$.
% 		\item Prices of leftover are zero: $p^*(\Theta\setminus (\cup_i \Theta_i)) = 0$.
% 	\end{enumerate}
% 	\label{cor:check-pure-alloc-ME}
% \end{corollary}

\paragraph{Fairness and efficiency guarantees.} 
Let $x\in (L^\infty(\Theta)_+)^n$, $\sum_i x_i \leq \ones$ be an allocation. It is (strongly) \emph{Pareto optimal} if there does \emph{not} exist $\tilde{x}\in (L^\infty(\Theta)_+)^n$, $\sum_i \tilde{x}_i \leq \ones$ such that $\langle v_i, \tilde{x}_i \rangle \geq \langle v_i, x_i \rangle$ for all $i$ and the inequality is strict for at least one $i$ \citep{cohler2011optimal}. 
It is \emph{envy-free} (in a budget-weighted sense) if 
$\frac{1}{B_i}\langle v_i, x_i \rangle \geq \frac{1}{B_j}\langle v_i, x_j\rangle$
for any $j\neq i$ \citep{nisan2007algorithmic,kroer2019computing}. 
When all $B_i$ are equal, this is sometimes referred to as being ``equitable'' \citep{weller1985fair}. 
It is \emph{proportional} if $\langle v_i, x_i \rangle \geq \frac{B_i}{\|B\|_1} v_i(\Theta)$ for all $i$, that is, each buyer gets at least the utility of its \emph{proportional share} allocation, 
$x^{\rm PS} := \frac{B_i}{\|B\|_1} \ones$. 
Similar to the finite-dimensional case, market equilibria in infinite-dimensional Fisher markets also exhibit these properties.
\begin{theorem}
	Let $(x^*, p^*)$ be a ME. Then, $x^*$ is Pareto optimal, envy-free and proportional. \label{thm:me-is-pareto-ef-prop}
\end{theorem}

\paragraph{ME as generalized fair division.}
By Theorem \ref{thm:me-is-pareto-ef-prop}, a pure ME $\{\Theta_i\}$ under uniform budgets ($B_i = 1/n$) is a fair division in the sense of \citet{weller1985fair}, that is, a Pareto optimal and envy-free division (into a.e.-disjoint measurable subsets) of $\Theta$. 
Furthermore, \citep[\S 3]{weller1985fair} shows that, there exist equilibrium prices $p^*$ such that 
(i) $p^*(\Theta_i) = 1/n$ for all $i$, 
(ii) $v_i(\Theta_i) \geq v_i(A)$ for any measurable set $A \subseteq \Theta$ such that $p^*(A) \leq 1/n$ and
(iii) for any measurable set $A\subseteq \Theta$, $p^*(A) = \frac{1}{n}\sum_i \frac{v_i(A\cap \Theta_i)}{v_i(\Theta_i)}$.
Utilizing our results, when $B_i = 1/n$, and $\{\Theta_i\}$ is a pure ME, property (i) above is a special case of $\langle p^*, x^*_i \rangle = B_i$ in Theorem~\ref{thm:eg-gives-me} (with $x^*_i = \ones_{\Theta_i}$); property (ii) above follows from the ME property $x^*_i \in D_i(p^*)$; property (iii) is a special case of Part \ref{item:thm-eg-pure-solution-gives-p*} in Corollary \ref{cor:eg-pure-solution-gives-strong-duality}, since $\beta^*_i = \frac{B_i}{u^*_i} = \frac{1}{n} \cdot \frac{1}{v_i(\Theta)}$. Hence, ME under a continuum of items can be viewed as generalized fair division.

% The following corollary formalizes this connection, where setting $B_i = 1/n$ precisely recovers the price characterization in \citep[\S 3]{weller1985fair}.
% In our case, assuming each buyer has budget $B_i$ such that $\|B\|_1 = 1$ (w.l.o.g.), a pure equilibrium allocations $\{\Theta_i\}$ and the a.e.-unique equilibrium prices $p^*$ 

% Setting $B_i = 1/n$ in the above recovers the characterizations of equilibrium prices in \cite[\S 3]{weller1985fair}. Specifically,  \eqref{eq:p*-decomposition-general-Bi} reduces to \eqref{eq:p*-decomposition-1/n-weller}, since $\beta^*_i = \frac{B_i}{u^*_i} = \frac{1}{n} \cdot \frac{1}{v_i(\Theta_i)}$ by Lemma \ref{lemma:weak-duality}. 
% \begin{corollary} 
% 	Let $\{\Theta_i\}$ be a pure equilibrium allocation and $\beta^* = \frac{B_i}{u^*_i}$.
% 	Then, the price $p^* = \max_i \beta^*_i v_i$ (which equals $\beta^*_i v_i$ a.e. on each buyer's allocated set $\Theta_i$) satisfies the following:
% 	\begin{itemize}
% 		\item $p^*(\Theta_i) = B_i$ for all $i$.
		
% 	\end{itemize}
% 	\label{cor:Bi-fair-division-weller}
% \end{corollary}

% \paragraph{Bounds on equilibrium quantities.} 
% Using the KKT condition $u^* = \frac{B_i}{\beta^*_i}$ (Lemma \ref{lemma:weak-duality}) and an equilibrium allocation being proportional (Theorem \ref{thm:me-is-pareto-ef-prop}), we can easily establish upper and lower bounds on equilibrium quantities. These bounds will be useful in subsequent convergence analysis of stochastic optimization in \S \ref{sec:sda}. 
% Similar bounds hold in the finite-dimensional case~\citep{gao2020first}. 
% Recall that we assume $v_i(\Theta) = 1$ and $\|B\|_1 = 1$ w.l.o.g.
% \begin{lemma}
% 	For any ME $(x^*, p^*)$, we have $p^*(\Theta) = 1$. 
% 	Furthermore, $B_i \leq u^*_i = \langle v_i, x^*_i\rangle \leq 1$ and hence $ B_i \leq \beta^*_i := \frac{B_i}{u^*_i} \leq 1$ for all $i$.
% 	\label{lemma:equi-bounds}
% \end{lemma}
\paragraph{Discrete and other measurable item spaces.} It can be easily verified that all the above results, except the existence of a pure ME, hold when $\Theta$ is a finite measure space (w.r.t. a measure $\mu$), including the classical case of a finite item set $\Theta$. To see this, note that Theorem~~\ref{thm:eg-gives-me} is based on Lemma~\ref{lemma:U-convex-compact}, which is a consequence of \citet[Theorems~1 and 4]{dvoretzky1951relations}. 
For a general finite measure space (where $\mu$ may not be \emph{atomless} like the Lebesgue measure), only \citet[Theorem~1]{dvoretzky1951relations} holds and hence Lemma~1 partially holds ($U$ is convex and compact, but $U'\neq U$ in general); nevertheless, this is sufficient to show the existence of ME and the convex programs capturing them. 

% \paragraph{Special cases of discrete item spaces $\Theta$.}
% All of the theory developed so far, except the existence of a \emph{pure} equilibrium allocation, holds when $\Theta$ is discrete (finite or \emph{countably} infinite).
% This is because, given a discrete $\Theta$, the set of feasible utilities $U = U(v, \Theta)$ defined in \eqref{eq:def-U-U(v,Theta)} is still closed and convex; however, a pure optimal solution of \eqref{eq:eg-primal} may not exist, as it requires $v_i$ to be atomless.
% % Hence, existence of a pure equilibrium allocation (or, equivalently, a pure optimal solution of \eqref{eq:eg-primal}) is not guaranteed. 
% We give more details for the cases of finite and countably infinite item spaces below.
% \begin{itemize}
% 	\item For a countably infinite $\Theta$, we can w.l.o.g. assume $\Theta = \mathbb{N}$ and $\mu(\Theta) = \sum_{\theta\in \Theta} \mu(\theta) = 1$. All subsets $A\subseteq \Theta$ are measurable with measure $\mu(A) = \sum_{\theta\in A} \mu(\theta)$. 
% 	In this case, a buyer's valuation is a nonnegative summable sequence $v_i \in \ell^1(\Theta)_+$, i.e., $\|v_i\| = \sum_{\theta\in \Theta} v_i(\theta) <\infty$ (we can assume $\|v_i\| = 1$ w.l.o.g., as discussed in \S \ref{sec:inf-dim-fim-setup}). A buyer's allocation is $x_i \in \ell^\infty(\Theta)_+$, i.e., $\sup_{\theta\in \Theta} x_i(\theta) < \infty$. 
% 	\item For a finite item space $\Theta = [m]$, we can take $\mu(A) := |A|/m$ for all $A\subseteq \Theta$. Buyers' valuations $v_i$ and allocations $x_i$ are nonnegative $m$-dimensional vectors, with $\langle v_i, x_i\rangle = \sum_{j\in [m]} v_{ij} x_{ij}$. 
% 	Here, to align with the normalization in \S \ref{sec:inf-dim-fim-setup}, we can set the supply of each item $j\in [m]$ to be $s_j = 1/m$ so that the total supply of all items is $\mu(\Theta) = \sum_{j\in [m]} s_j = 1$. To ensure $\|v_i\| = v_i(\Theta) = \sum_{j\in [m]} s_j v_{ij} = 1$ (where the norm is w.r.t. the Banach space with measure $\mu$), which means $v_i(\Theta) = \sum_{j\in [m]} v_{ij} s_j = \frac{1}{m}\|v_i\|_1 = 1$, i.e., $\|v_i\|_1 = m$ (where the norm is the usual Euclidean $1$-norm). 
% \end{itemize}