% \section{Introduction}
% Market equilibrium (ME) is a classical concept from economics, where the goal is to find an allocation of a set of items to a set of buyers, as well as corresponding prices, such that the market clears. 
% One of the simplest equilibrium models is the (finite-dimensional) linear \emph{Fisher market}. A Fisher market consists of a set of $n$ buyers and $m$ divisible items, where the utility for a buyer is linear in their allocation.
% % The concept of a market equilibrium (ME) in a (finite-dimensional) linear Fisher market has been studied extensively
% % has been defined under a (finite-dimensional) linear Fisher market consists of $n$ buyers and $m$ items 
% Each buyer $i$ has a budget $B_i$ and valuation $v_{ij}$ for each item $j$. A ME consists of an allocation (of items to buyers) and prices (of items) such that (i) each buyer receives a bundle of items that maximizes their utility subject to their budget constraint, and (ii) the market clears (all items such that $p_j>0$ are exactly allocated).
% In spite of its simplicity, this model has several applications. For example, the most well-known application is in the \emph{competitive equilibrium from equal incomes} (CEEI), where  $m$ items are to be fairly divided among $n$ agents. By giving each agent one unit of faux currency, the allocation from the resulting ME can be used as a fair division. This approach guarantees several fairness desiderata, such as envy-freeness and proportionality. Beyond fair division, linear Fisher markets also find applications in large-scale ad markets \citep{conitzer2018multiplicative,conitzer2019pacing} and fair recommender systems \citep{kroer2019computing,kroer2019scalable}.

% For the case of finite-dimensional linear Fisher markets, the Eisenberg-Gale convex program computes a market equilibrium \citep{eisenberg1959consensus,eisenberg1961aggregation,jain2010eisenberg,nisan2007algorithmic,cole2017convex}.
% However, in settings like Internet ad markets and fair recommender systems, the number of items is often huge~\citep{kroer2019computing,kroer2019scalable}, if not infinite or even uncountable~\citep{balseiro2015repeated}. For example, each item can be characterized by a set of features.
% In that case, a natural model for an extremely-large market, such as Internet ad auctions or recommender systems, is to assume that the items are drawn from some underlying distribution over a compact set of possible features.
% Motivated by such settings where the item space is most easily modeled as a continuum, we study Fisher markets and ME for a continuum of items.

% %[More motivation? Infinitely many items related to ad allocation and recommender system; First-price pacing equilibrium (ad auction); market abstraction (fair recommender system); mention continuous models for ad allocation (approximating a huge number of items, )]

% A problem closely related to our infinite-dimensional Fisher-market setting is the \textit{cake-cutting} or \textit{fair division} problem. There, the goal is to efficiently partition a ``cake'' -- often modeled as a compact measurable space, or simply the unit interval $[0,1]$ -- among $n$ agents so that certain fairness and efficiency properties are satisfied \citep{weller1985fair,brams1996fair,cohler2011optimal,procaccia2013cake,cohler2011optimal,brams2012maxsum,chen2013truth,aziz2014cake,aziz2016discrete,legut2017optimal,legut2020obtain}. See \citep{procaccia2014cake} for a survey for the various problem setups, algorithms and complexity results. \citet{weller1985fair} shows the existence of a fair allocation, that is, a measurable division of a measurable space satisfying weak Pareto optimality and envy freeness. As will be seen shortly, when all buyers have the same budget, our definition of a \textit{pure} ME, i.e., where the allocation consist of indicator functions of a.e.-disjoint measurable sets, is in fact equivalent to this notion of fair division, that is, a Pareto optimal, envy-free division (also see, e.g., \citep{cohler2011optimal,chen2013truth}). 
% Additionally, we also give an explicit characterization of the unique equilibrium prices based on a pure equilibrium allocation under arbitrary budgets. 
% This generalizes the result of \citet{weller1985fair}, which only holds for uniform budgets.

% \yuan{Under piecewise \emph{constant} valuations over the $[0,1]$ interval, the equivalence of fair division and market equilibrium in certain setups has been discovered and utilized in the design of cake-cutting algorithms \citep{brams2012maxsum,aziz2014cake}. For example, \citet{aziz2014cake} shows that the special case with piecewise constant valuations can be (easily) reduced to a finite-dimensional Fisher market and hence captured by the classical Eisenberg-Gale framework. Our infinite-dimensional convex optimization characterization extends this connection from piecewise constant valuations to arbitrary valuations in the $L^1$ function space: we propose Eisenberg-Gale-type convex programs that characterize \textit{all} ME, which include pure ME that, under uniform budgets, correspond to fair divisions.}

% \yuan{The more general case with piecewise liner valuations has been considered, although with different fairness and efficiency objectives, and is deemed challenging \citep[\S 4]{cohler2011optimal}. Different from the piecewise constant case, here, one cannot cut the unit interval \emph{a priori} based on breakpoints of the pieces of buyers' valuations. Instead, as we will see, the correct such ``cuts'' inevitably depend on the equilibrium utility prices (price per unit utility) of each buyer.
% As a concrete application of our framework, we show that we can efficiently compute equilibrium quantities via convex conic optimization under piecewise linear valuations. 
% To this end, in addition to the convex optimization characterization, another key result is that, given linear buyer valuations over a closed interval, the set of possible utilities spanned by all possible allocations of the item space can be described by a small number of linear and quadratic constraints. This result may be generalized to other classes of buyer valuations and thus be of independent interest.}
% % the special case with piecewise \emph{constant} valuations can be reduced to a finite-dimensional Fisher market and hance captured by the classical Eisenberg-Gale framework \cite{aziz2014cake}. The case with 
% %A recent related work is that of \citet{legut2017optimal}, which proposes optimization formulations for ``optimal'' fair division under linear utilities. This work considers optimality and fairness notions that are different from the more commonly used one we adopt.   Furthermore, the optimization formulation is non-convex, and thus does not lead to efficient computability, or even guarantee existence. 
% %Their formulation also does not lead to a market equilibrium.
% %\ck{Alternative stuff it would be good to say here. Please check if these are all true and then incorporate above:
% %  Furthermore, the optimization formulation is non-convex, and thus does not lead to efficient computability, or even guarantee existence. 
% %  Their formulation also does not lead to market equilibrium properties.
% %}

% % \yuan{More concrete: there, \cite{aziz2014cake} reduces the problem into finite-dim Fisher market; that does not help here for the ${\rm p.w.l.}$ case. Also cite cake-cutting papers mentioning hardness of ${\rm p.w.l.}$ valuations \cite{procaccia2014cake}}

% \subsection{Summary of contributions}
% \paragraph{Infinite-dimensional Fisher markets and equilibria.} First, we propose the notion of a market equilibrium (ME) for an infinite-dimensional Fisher market with $n$ buyers and a continuum of items $\Theta$. 
% Here, buyers' valuations are nonnegative $L^1$ functions on $\Theta$. Buyers' allocations are nonnegative $L^\infty$ functions on $\Theta$. 
% A special case is \emph{pure} allocations, where buyers get a.e.-disjoint measurable sets of items. 
% % We sometimes restrict ourselves to \emph{pure} allocations

% \paragraph{Market equilibrium and convex optimization duality.}
% We then give two infinite-dimensional convex programs over Banach spaces of measurable functions on $\Theta$, \eqref{eq:eg-primal} and \eqref{eq:eg-dual-beta-p}, generalizing the EG convex program and its dual for finite-dimensional Fisher markets. 
% For the new convex programs, we first establish existence of optimal solutions (Parts \ref{item:eg-primal-attain} and \ref{item:eg-dual-p-beta-attain} of Theorem \ref{thm:eg-equi-opt-combined}). 
% Due to the lack of a compatible constraint qualification, general duality theory does not apply to these convex programs. 
% Instead, we establish various duality properties directly through nonstandard arguments (Lemma \ref{lemma:weak-duality} and Theorem \ref{thm:eg-pure-solution-gives-strong-duality}).
% Based on these duality properties and the existence of a minimizer in the ``primal'' convex program \eqref{eq:eg-primal}, we show that a pair of allocations and prices is a ME \textit{if and only if} they are optimal solutions to the convex programs (Part \ref{item:eg-equi-iff-opt} of Theorem \ref{thm:eg-equi-opt-combined} and Theorem \ref{thm:eg-gives-me}). Furthermore, since \eqref{eq:eg-primal} exhibits a \emph{pure} optimal solution, i.e., buyers get disjoint subsets of items (Part \ref{item:eg-primal-attain} of Theorem \ref{thm:eg-equi-opt-combined}), we conclude that there exists a pure equilibrium allocation, i.e., a \textit{division} (module zero-value items) of the item space.

% \paragraph{Properties of a market equilibrium.} 
% Based on the above results, we further show that a ME under the infinite-dimensional Fisher market satisfies (budget-weighted) proportionality, Pareto optimality and envy-freeness. Our results on the existence of ME and its fairness properties can be viewed as generalizations of those in \citep{weller1985fair}, in which every buyer (agent) has the same budget. 

% All of the above results, except the existence of a pure equilibrium allocation, hold when the item space $\Theta$ is discrete (finite or countably infinite).

% \paragraph{Tractable reformulation under piecewise linear valuations.} 
% When the item space is a closed interval (e.g., $[0,1]$) and buyers have piecewise linear utilities, we show that equilibrium allocations can be computed efficiently via solving a convex conic reformulation of the infinite-dimensional Eisenberg-Gale-type convex program \eqref{eq:ql-eg-primal}.
% This gives an efficient algorithm for computing a fair division under piecewise linear valuations, the first polynomial-time algorithm for this challenging problem to the best of our knowledge.
% The key in the reformulation is to show that, for linear valuations on a closed interval, we can characterize the set of feasible utilities that can be attained by all feasible allocations by a small number of linear and quadratic constraints with a few auxiliary variables.

% \paragraph{Stochastic optimization for general valuations.}
% Finally, for more general buyer valuations or a huge number of buyers, we propose solving the finite-dimensional convex program \eqref{eq:eg-dual-beta-1} for equilibrium utility prices using the stochastic dual averaging algorithm (SDA) and establish its convergence guarantees (Theorem \ref{thm:sda-conv}).


\section{Introduction}
Market equilibrium (ME) is a classical concept from economics, where the goal is to find an allocation of a set of items to a set of buyers, as well as corresponding prices, such that the market clears. 
One of the simplest equilibrium models is the (finite-dimensional) linear \emph{Fisher market}. A Fisher market consists of a set of $n$ buyers and $m$ divisible items, where the utility for a buyer is linear in their allocation.
% The concept of a market equilibrium (ME) in a (finite-dimensional) linear Fisher market has been studied extensively
% has been defined under a (finite-dimensional) linear Fisher market consists of $n$ buyers and $m$ items 
Each buyer $i$ has a budget $B_i$ and valuation $v_{ij}$ for each item $j$. A ME consists of an allocation (of items to buyers) and prices (of items) such that (i) each buyer receives a bundle of items that maximizes their utility subject to their budget constraint, and (ii) the market clears (all items such that $p_j>0$ are exactly allocated).
In spite of its simplicity, this model has several applications. Perhaps one of the most celebrated examples is the \emph{competitive equilibrium from equal incomes} (CEEI), where  $m$ items are to be fairly divided among $n$ agents. By giving each agent one unit of faux currency, the allocation from the resulting ME can be used as a fair division. This approach guarantees several fairness desiderata, such as envy-freeness and proportionality. Beyond fair division, linear Fisher markets also find applications in large-scale ad markets \citep{conitzer2018multiplicative,conitzer2019pacing} and fair recommender systems \citep{kroer2019computing,kroer2019scalable}.

For the case of finite-dimensional linear Fisher markets, the Eisenberg-Gale convex program computes a market equilibrium \citep{eisenberg1959consensus,eisenberg1961aggregation}.
However, in settings like Internet ad markets and fair recommender systems, the number of items is often huge~\citep{kroer2019computing,kroer2019scalable}, if not infinite or even uncountable~\citep{balseiro2015repeated}. For example, each item can be characterized by a set of features.
In that case, a natural model for an extremely large market, such as Internet ad auctions or recommender systems, is to assume that the items are drawn from some underlying distribution over a compact set of possible feature vectors.

Motivated by settings such as the above, where the item space is most easily modeled as a continuum, we study Fisher markets and its equilibria for a continuum of items.
% \ck{
    While equilibrium computation for finite-dimensional linear Fisher markets is well understood, nothing is known about computation of its infinite-dimensional analogue. We rectify this issue by developing infinite-dimensional convex programs over Banach spaces that generalize the Eisenberg-Gale convex program and its dual. We show that these convex programs lead to market equilibria, and give scalable first-order methods for solving the convex programs.
    % }

%[More motivation? Infinitely many items related to ad allocation and recommender system; First-price pacing equilibrium (ad auction); market abstraction (fair recommender system); mention continuous models for ad allocation (approximating a huge number of items, )]

A problem closely related to our infinite-dimensional Fisher-market setting is the \textit{cake-cutting} or \textit{fair division} problem. 
There, the goal is to efficiently partition a ``cake''---often modeled as a compact measurable space, or simply the unit interval $[0,1]$---among $n$ agents so that certain fairness and efficiency properties are satisfied \citep{weller1985fair,brams1996fair,cohler2011optimal,procaccia2013cake,cohler2011optimal,brams2012maxsum,chen2013truth,aziz2014cake,aziz2016discrete,deng2012algorithmic}.
% \ck{
Focusing on the case of finding a division of a measurable space satisfying weak Pareto optimality and envy freeness, \citet{weller1985fair} shows the existence of a fair division.
When all buyers have the same budget, our definition of a \textit{pure} ME, i.e., where the allocation consist of indicator functions of a.e.-disjoint measurable sets, is equivalent to this notion of fair division.
% (see also \citet{cohler2011optimal,chen2013truth}). 
Thus, our convex programs yield solutions to the fair division setting of \citet{weller1985fair}.
% }
Additionally, we also give an explicit characterization of the unique equilibrium prices based on a pure equilibrium allocation under arbitrary budgets. 
This generalizes the result of \citet{weller1985fair}, which only holds for uniform budgets.
% See \citep{procaccia2014cake} for a survey of various other problem setups and algorithms. 

Under piecewise \emph{constant} valuations over the $[0,1]$ interval, the equivalence of fair division and market equilibrium in certain setups has been discovered and utilized in the design of cake-cutting algorithms \citep{brams2012maxsum,aziz2014cake}. 
For example, \citet{aziz2014cake} show that the special case with piecewise constant valuations can be (easily) reduced to a finite-dimensional Fisher market and hence captured by the classical Eisenberg-Gale framework. 
Our infinite-dimensional convex optimization characterization extends this connection from piecewise constant valuations to arbitrary valuations in the $L^1$ function space: we propose Eisenberg-Gale-type convex programs that characterize \textit{all} ME. This includes pure ME which, under uniform budgets, correspond to fair divisions. 

% \ck{
Beyond piecewise constant valuations, piecewise linear valuations have also been considered in fair division, although with different fairness and efficiency objectives. This setting is considerably more challenging, and e.g. \citet[\S 4]{cohler2011optimal} focus on the case of two agents. Unlike the piecewise constant case, one cannot cut the unit interval \emph{a priori} based on breakpoints of the pieces of buyers' valuations. Instead, as we will see, the correct such ``cuts'' inevitably depend on the equilibrium utility prices (price per unit utility) of each buyer.
Nonetheless, we will show that in the piecewise linear case, it is possible to reformulate our general convex program as a finite-dimensional convex conic program involving second-order cones and exponential cones. 
We leverage this reformulation to give a polynomial-time procedure for computing a fair division for piecewise linear utilities in complete generality, for any number of agents.
In addition to being polynomial-time computable in theory, our conic reformulation is also highly efficient numerically: it can be written as a sparse conic program that can be solved with standard convex optimization software.
% As a concrete application of our framework, we show that we can efficiently compute equilibrium quantities via convex conic optimization under piecewise linear valuations. 
A key part of our finite-dimensional conic reformulation is to show that, given linear buyer valuations over a closed interval, the set of utilities attainable by feasible allocations of the item space can be described by a small number of linear and quadratic constraints. This result may be generalized to other classes of buyer valuations and may thus be of independent interest.
% }

% \paragraph{Other related work.} 
% \yuan{Cite Cheung \& Cole on dynamics. Cite Garg \& Bichler on different utilities. \texttt{https://dl.acm.org/doi/pdf/10.1145/3340234}} 

% the special case with piecewise \emph{constant} valuations can be reduced to a finite-dimensional Fisher market and hence captured by the classical Eisenberg-Gale framework \cite{aziz2014cake}. The case with 
%A recent related work is that of \citet{legut2017optimal}, which proposes optimization formulations for ``optimal'' fair division under linear utilities. This work considers optimality and fairness notions that are different from the more commonly used one we adopt.   Furthermore, the optimization formulation is non-convex, and thus does not lead to efficient computability, or even guarantee existence. 
%Their formulation also does not lead to a market equilibrium.
%\ck{Alternative stuff it would be good to say here. Please check if these are all true and then incorporate above:
%  Furthermore, the optimization formulation is non-convex, and thus does not lead to efficient computability, or even guarantee existence. 
%  Their formulation also does not lead to market equilibrium properties.
%}

% \yuan{More concrete: there, \cite{aziz2014cake} reduces the problem into finite-dim Fisher market; that does not help here for the ${\rm p.w.l.}$ case. Also cite cake-cutting papers mentioning hardness of ${\rm p.w.l.}$ valuations \cite{procaccia2014cake}}

\subsection{Summary of contributions}
\paragraph{Infinite-dimensional Fisher markets and equilibria.} First, we propose the notion of a market equilibrium (ME) for an infinite-dimensional Fisher market with $n$ buyers and a continuum of items $\Theta\subseteq \RR^d$ (the case of $\Theta$ being a general finite measure space will be discussed at the end of \S \ref{sec:equi-and-dual}).
Here, buyers' valuations and allocations are nonnegative $L^1$ and $L^\infty$ functions on $\Theta$, respectively. 
A special case is \emph{pure} allocations, where buyers get a.e.-disjoint measurable sets of items. 
% We sometimes restrict ourselves to \emph{pure} allocations

\paragraph{Market equilibrium and convex optimization duality.}
We then give two infinite-dimensional convex programs over Banach spaces of measurable functions on $\Theta$, which generalize the EG convex program and its dual for finite-dimensional Fisher markets, and establish the existence of optimal solutions.
% (Parts \ref{item:eg-primal-attain} and \ref{item:eg-dual-p-beta-attain} of Theorem \ref{thm:eg-equi-opt-combined}). 
Due to the lack of a compatible constraint qualification, general duality theory does not apply to these convex programs. 
Instead, we establish various duality properties directly through nonstandard arguments. %(Lemma \ref{lemma:weak-duality} and Theorem \ref{thm:eg-pure-solution-gives-strong-duality}).
Based on these duality properties and the existence of a minimizer in the ``primal'' convex program, we show that a pair of allocations and prices is a ME \textit{if and only if} they are optimal solutions to the convex programs.
% (Part \ref{item:eg-equi-iff-opt} of Theorem \ref{thm:eg-equi-opt-combined} and Theorem \ref{thm:eg-gives-me}). 
Furthermore, we show that the primal convex program admits a \emph{pure} optimal solution, meaning that buyers get disjoint subsets of items.
% (Part \ref{item:eg-primal-attain} of Theorem \ref{thm:eg-equi-opt-combined}), 
and we conclude that there exists a pure equilibrium allocation, i.e., a \textit{division} (modulo zero-value items) of the item space.

\paragraph{Properties of a market equilibrium.} 
Based on the above results, we further show that a ME under the infinite-dimensional Fisher market satisfies (budget-weighted) proportionality, Pareto optimality and budget-weighted envy-freeness. Our results on the existence of ME and its fairness properties can be viewed as generalizations of those in \citet{weller1985fair}, in which every buyer (agent) has the same budget.

\paragraph{Tractable reformulation under piecewise linear valuations.} 
When the item space is a closed interval (e.g., $[0,1]$) and buyers have piecewise linear valuations, we show that equilibrium allocations can be computed efficiently via solving a convex conic reformulation of the infinite-dimensional Eisenberg-Gale-type convex program. %\eqref{eq:ql-eg-primal}.
This gives an efficient algorithm for computing a fair division under piecewise linear valuations, the first polynomial-time algorithm for this challenging problem to the best of our knowledge.
The key in the reformulation is to show that, for linear valuations on a closed interval, the set of feasible utilities spanned by all feasible allocations can be described by a small number of linear and quadratic constraints with a few auxiliary variables.

\paragraph{Stochastic optimization and quasilinear extensions.} 
We briefly discuss how more general valuations can be handled via stochastic optimization and how the existence and characterization results extend to the case of quasilinear utilities.

%%%%%%%%%%%%%%%%%%%%%%%%%%%%%%%%%%%%%%%%%%%%%%%%%%%%%%%%%%%%%%%
%%%%%%%%%%%%%%%%%%%%%%%%%%%%%%%%%%%%%%%%%%%%%%%%%%%%%%%%%%%%%%%
% \paragraph{Stochastic optimization for general valuations.}
% For more general buyer valuations or a huge number of buyers, we propose solving a finite-dimensional convex reformulation of the dual of the infinite-dimensional EG for equilibrium utility prices using the stochastic dual averaging algorithm (SDA) and establish its convergence guarantees.% (Theorem \ref{thm:sda-conv}).

%%%%%%%%%%%%%%%%%%%%%%%%%%%%%%%%%%%%%%%%%%%%%%%%%%%%%%%%%%%%%%%
%%%%%%%%%%%%%%%%%%%%%%%%%%%%%%%%%%%%%%%%%%%%%%%%%%%%%%%%%%%%%%%
% \paragraph{Extension to quasilinear utilities.} 
% Most of the above results easily extend to the setting where each buyer has a quasilinear utility. 
% Specifically, we show that, in this case, a different pair of infinite-dimensional convex programs exhibit optimal solutions that correspond to quasilinear market equilibria. 
% The convex conic reformulation can also be modified easily to capture pure quasilinear equilibrium allocations under piecewise linear valuations. 
% Finally, SDA can also be easily modified to compute equilibrium utility prices in the quasilinear setting.

%%%%%%%%%%%%%%%%%%%%%%%%%%%%%%%%%%%%%%%%%%%%%%%%%%%%%%%%%%%%%%%
%%%%%%%%%%%%%%%%%%%%%%%%%%%%%%%%%%%%%%%%%%%%%%%%%%%%%%%%%%%%%%%
% \subsection{Related work}
% In addition to the aforementioned works, we briefly review other recent works on market equilibrium computation and fair division.
% \paragraph{Market equilibrium computation.} 
% For the classical Fisher market setting with finitely many items, there is a large literature on equilibrium computation algorithms, some based on solving new equilibrium-capturing convex programs. For example, 
% \citet{devanur2008market} established the first polynomial-time algorithm for exact equilibrium computation for a finite-dimensional linear Fisher market, based on a primal-dual algorithm for solving the Eisenberg-Gale convex program. 
% \citet{zhang2011proportional} proposed a distributed dynamics that converges to an equilibrium, which, as later analyzed in \citet{birnbaum2011distributed}, is in fact a first-order optimization algorithm applied to a specific convex program due to \citet{shmyrev2009algorithm}.
% \citet{cheung2019tatonnement} studied t\^{a}tonnement dynamics and show its equivalence to gradient descent on Eisenberg-Gale-type convex programs under the more general class of CES utilities. \citet{gao2020first} studied first-order methods based on (old and new) convex programs for Fisher market equilibria under commonly used utilities. \citet{bei2019earning} studied earning and utility limits in markets with linear and spending-constraint utilities, and proposed a polynomial-time algorithm for computing an equilibrium. 
% %Pivoting algorithms of combinatorial nature have also been studied, for the more general Arrow–Debreu exchange market and piecewise linear separable concave utilities \citep{garg2013towards,garg2015complementary}.

%%%%%%%%%%%%%%%%%%%%%%%%%%%%%%%%%%%%%%%%%%%%%%%%%%%%%%%%%%%%%%%
%%%%%%%%%%%%%%%%%%%%%%%%%%%%%%%%%%%%%%%%%%%%%%%%%%%%%%%%%%%%%%%
% \paragraph{Fair division.}
% As stated previously, a Fisher market equilibrium on finitely many divisible items is known to be a strong fair division approach. 
% There is also a literature on fair division of \emph{indivisible} items via maximizing the Nash social welfare (NSW); the discrete analogue of the Eisenberg-Gale program.
% This was started by \citet{caragiannis2016unreasonable}, who showed that the maximum NSW solution provides fairness guarantees in the indivisible divisible case as well, and proposed a practical algorithm based on mixed-integer linear programming.
% There are also several works on approximation algorithms for this settings, see e.g. \citet{garg2018approximating,barman2018finding}.
% Interestingly, we will show that our continuum setting allows us to construct allocations via convex programming, even for the indivisible setting.